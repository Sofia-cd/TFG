\documentclass[12pt,twoside,onecolumn,a4paper]{tesis}
\usepackage[T1]{fontenc}
\usepackage[latin1]{inputenc}
\usepackage[english,spanish]{babel}
\usepackage[dvips]{graphicx}
\usepackage[small]{subfigure}
\usepackage{fancyheadings}
\usepackage{amsfonts}
\usepackage{amssymb}
\usepackage{multirow}

\pagestyle{fancy}
\bibliographystyle{plain}
\renewcommand{\chaptermark}[1]{\markboth{\chaptername\ \thechapter. #1}{}}
\renewcommand{\sectionmark}[1]{\markright{\thesection. #1}}
\renewcommand{\baselinestretch}{1.1}
\lhead[\fancyplain{}{\thepage}]{\fancyplain{}{\rightmark}}
\rhead[\fancyplain{}{\leftmark}]{\fancyplain{}{\thepage}}
\cfoot{}


\renewcommand{\theenumi}{\arabic{enumi}.}
\renewcommand{\theenumii}{\theenumi\arabic{enumii}.}
\renewcommand{\theenumiii}{\theenumii\arabic{enumiii}.}
\renewcommand{\labelenumi}{\theenumi}
\renewcommand{\labelenumii}{\theenumii}
\renewcommand{\labelenumiii}{\theenumiii}

\oddsidemargin=2.5 cm
\marginparwidth=4 cm
\linespread{1.3}


%funcion para compilar solo una parte. Un solo fichero.
%\includeonly{intro,grid}


\begin{document}
\pagestyle{empty}
\centerline{ }

\begin{center}

{\Large \bf FACULTADE DE INFORM�TICA}

\vspace{0.5cm}

 \emph{Departamento de Electr�nica e Sistemas}

\vspace{0.25cm}

\begin{center}
TRABAJO FIN DE GRADO
GRADO EN INGENIER�A INFORM�TICA
MENCI�N EN INGENIER�A DEL SOFTWARE
\end{center}

\vspace{1.5cm}

 \Huge \bf Paralelizaci\'{o}n de un algoritmo de optimizaci\'{o}n de colonia de hormigas aplicado al problema del viajante.
 \end{center}

\vspace{2cm}

\vspace{0.25cm} \hfill {\textbf{    Autor:} Sof�a Castro Dom�nguez}

\vspace{0.25cm} \hfill {\textbf{    Tutor:}  Patricia Gonz�lez G�mez}


\vspace{0.5cm} \hfill {A Coru�a, Junio de 2021}

\cleardoublepage
\pagestyle{empty}
\chapter*{Resumen}

Este Trabajo Fin de Grado tiene como objetivo estudiar la estructura básica del algoritmo {\em Optimización de Colonia de Hormigas (Ant Colony Optimization - ACO) }para evaluar la efectividad y eficiencia de diferentes estrategias de parelización y su combinación.

\section*{Palabras clave}
Computaci\'on Grid, ...

\cleardoublepage \pagenumbering{roman} \pagestyle{fancy}
\tableofcontents \listoftables \listoffigures \cleardoublepage
\pagenumbering{arabic} \pagestyle{fancy}

%%%%% Aqui van los capitulos. Cada uno es un fichero tex diferente

2\chapter{Introducci\'on}\label{cap:intro}

\section{Motivaci\'on}
En la actualidad hay multitud de \'ambitos en los que los problemas NP-complejos cobran vital importancia y son estrat\'egicos para el sector. Dichos campos abarcan desde la optimizaci\'on combinatoria, bioinform\'atica, telecomunicaciones hasta la econom\'ia, ingenier\'ia software, etc...\\
Tradicionalemente se utilizan dos enfoques principales para abordar estos problemas, los m\'etodos exactos y las metaheur\'isticas. Pese a que hoy en d\'ia las computadoras han evolucionado r\'apidamente y son capaces de procesar gran cantidad de operaciones por segundo, no se suele disponer de los recursos nescesarios, ya sean exigencias hardware como de tiempo, para resolver estos problemas empleando m\'etodos exactos. He aqu\'i el inter\'es de estudiar una metahur\'istica como es The Ant Colony Optimization method (ACO), con un \'exito especial, y tratar de incrementar las ventajas inherentes a las metaheur\'iticas paralelizando y consiguiendo soluciones de calidad en menos tiempo.

\cite{libro1}

\section{Objetivos}\label{cap:objetivos}
La finalidad de este TFG es analizar la estructura de los algotimos de optimizaci\'on combinatoria, como es el caso de ACO, para idear y estudiar las distintas posibilidades de paralelizaci\'on. Una vez implementadas las distintas propuestas paralelas se evaluar\'an los resultados, empleando el problema del viajante TSP. Poniendo en pr\'actica y ampliando, de esta manera, los conocimientos sobre la librer\'ia OpenMP y MPI. Adem\'as de conocer y utilizar un entorno HPC, que ser\'a donde se realizar\'an las pruebas.


\cleardoublepage
\chapter{Computaci\'on Grid}

\cleardoublepage
\chapter{Dise�o e implementaci�n de la herramienta}
En este cap�tulo se detalla c�mo se ha implementado la herramienta.

\cleardoublepage
\chapter{Uso}

\cleardoublepage
\chapter{Conclusiones y principales aportaciones}
El trabajo presentado pretende proporcionar ....
\cleardoublepage
\renewcommand\chaptername{Ap�ndice}
\appendix
\chapter{Ap\'endice si fuese necesario}

Aqu\'{i} se escribe el ap\'endice como un cap\'{i}tulo normal, si fuese necesario.


\cleardoublepage
\markboth{Bibliograf�a}{Bibliograf�a}
\addcontentsline{toc}{chapter}{Bibliograf�a}
\bibliography{biblio}

\end{document}

