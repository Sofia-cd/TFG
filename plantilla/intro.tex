\chapter{Introducci�n}\label{cap:intro}

%% El \label se usa para etiquetar un capitulo (o seccion, o figura) y luego referenciarlo a lo largo del texto sin tener que poner explicitamente su numero.

\section{Secci\'on}

\subsection{subapartado}

\subsubsection{subsubapartado}

\subsection*{subapartado sin numerar}

Para poner en {\bf negrilla} o bien \textbf{negrilla}.

Para poner en {\em cursiva} o bien \emph{cursiva}.

Para referenciar algo que hemos etiquetado antes, como este capitulo \ref{cap:intro}.

Para citar una fuente bibliografica \cite{libro1}. La bibliograf\'ia se introduce en el fichero biblio.bib. Hay que compilar este fichero para generar la bibliograf\'ia de la memoria.

Para poner una figura:

\begin{figure}[tbh]
\begin{center}
\includegraphics[width=0.9\textwidth]{imagenes/00.eps}
 \caption{Texto al pie de la figura.}
\label{fig:nombre}
\end{center}
\end{figure}

